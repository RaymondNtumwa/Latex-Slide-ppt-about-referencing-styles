\documentclass{beamer}
\usetheme{CambridgeUS}
\title{Descriptive Comparison between Referencing Styles: A Review about Harvard, IEEE, APA referencing styles. }
\author{Raymond Ntumwa \\ S24B23/101}
\date{\today}
\usepackage{graphicx} % For including images

\begin{document}

\begin{frame}
    \titlepage
\end{frame}

\begin{frame}
    \frametitle{Harvard Reference Style}
    \begin{block}{Definition}
    Harvard Referencing Style is one of the most commonly used referencing strategies in academic writing; -commonly used in UK universities 
    \end{block}
    It uses references in two places in a piece of writing: in text and in a reference list at the end; this means that the Author's name will appear in the text and in the references list. All the work in the reference list must be in the main.tex . Harvard Referencing gives the author's last name and year of work's publication and if we quote or paraphrase, we should also give a page number.
    \begin{exampleblock}{Example}
    Martinez, C. (2022). 'Urban migration patterns in South America'. Journal of Population Studies, 45(3), pp. 234-251.
        
    \end{exampleblock}
   % \tableofcontents
\end{frame}

%\section{Introduction}
\begin{frame}
    \frametitle{IEEE Reference Style}
    \begin{block}{Definition}
        IEEE referencing is a numerical style designed by the Institute of Electrical and Electronics Engineers that is  commonly used in technical fields. 
    \end{block}.
    Referencing in the IEEE style is a two-part process:
    \begin{itemize}
        \item A number in the text: a numerical reference in the text, relating to a numbered reference in the reference list. The citation number should be placed directly after the reference and should be included inside the punctuation within a sentence.
        \item Reference list: a complete list of all the cited references, numbered sequentially and with full bibliographic details.
    \end{itemize}
    \begin{exampleblock}{Example}
        [1] N. Shinohara, Wireless Power Transfer via Radiowaves. London: ISTE, 2014.
    \end{exampleblock}
\end{frame}

% \section{Main Ideas}
\begin{frame}
    \frametitle{APA Reference Style}
    \begin{block}{Definition}
        The American Psychological Association (APA) style is used in psychology, health and the social sciences. 
    \end{block}
    In an APA-style paper, you’ll identify the author and year of each source any time you
use it. That information directs readers to more detailed entries on a reference list at the paper’s end.
    \begin{exampleblock}{Example}
        Kumar, S., & Osei, K. (2023). Mental health interventions in university students. Journal of Clinical Psychology, 79(4), 1123-1140. https://doi.org/10.1002/jclp.23145
    \end{exampleblock}
\end{frame}

\end{document}